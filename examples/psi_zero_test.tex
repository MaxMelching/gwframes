\documentclass[
	preview,
	% tikz,  % In principle best thing you can do, but does not work with subfigures
	border=3pt,
]{standalone}


\usepackage{subfig}

\usepackage{cbc_frames_tikz}



\begin{document}
\begin{figure}
	\centering

	\subfloat[$\psi = 0$]{
		\begin{tikzpicture}[scale=0.75]
		    \drawframes[%
		        inclination=0,
		        polarization=0,
		        longascnodes=-90,
		        ra=0,
		        dec=90,
		        eccentricity=0.5,
		        binarydistance=1.5,
		        signalframehelperlines=false,
		        sourceframeaxes=false,
		        distancelabel=,
		        lineofsightlabel=,
		        earthtilt=24,
		    ];
		\end{tikzpicture}
	}
	%
	\subfloat[$\psi = \pi/4$]{
		\begin{tikzpicture}[scale=0.75]
		    \drawframes[%
		        inclination=0,
		        polarization=45,
		        longascnodes=-90,
		        ra=0,
		        dec=90,
		        eccentricity=0.5,
		        binarydistance=1.5,
		        signalframehelperlines=false,
		        sourceframeaxes=false,
		        distancelabel=,
		        lineofsightlabel=,
		    ];
		\end{tikzpicture}
	}
	
	\subfloat[$\psi = \pi/2$]{
		\begin{tikzpicture}[scale=0.75]
		    \drawframes[%
		        inclination=0,
		        polarization=90,
		        longascnodes=-90,
		        ra=0,
		        dec=90,
		        eccentricity=0.5,
		        binarydistance=1.5,
		        signalframehelperlines=false,
		        sourceframeaxes=false,
		        distancelabel=,
		        lineofsightlabel=,
		    ];
		\end{tikzpicture}
	}
	%
	\subfloat[$\psi = \pi$]{
		\begin{tikzpicture}[scale=0.75]
		    \drawframes[%
		        inclination=0,
		        polarization=180,
		        longascnodes=-90,
		        ra=0,
		        dec=90,
		        eccentricity=0.5,
		        binarydistance=1.5,
		        signalframehelperlines=false,
		        sourceframeaxes=false,
		        distancelabel=,
		        lineofsightlabel=,
		    ];
		\end{tikzpicture}
	}
	
	
	\caption{Impact Of Polarization Angle}
\end{figure}
\end{document}
